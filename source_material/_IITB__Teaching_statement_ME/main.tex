%%%%%%%%%%%%%%%%%%%%%%%%%%%%%%%%%%%%%%%%%
% "ModernCV" CV and Cover Letter
% LaTeX Template
% Version 1.1 (9/12/12)
%
% This template has been downloaded from:
% http://www.LaTeXTemplates.com
%
% Original author:
% Xavier Danaux (xdanaux@gmail.com)
%
% License:
% CC BY-NC-SA 3.0 (http://creativecommons.org/licenses/by-nc-sa/3.0/)
%
% Important note:
% This template requires the moderncv.cls and .sty files to be in the same
% directory as this .tex file. These files provide the resume style and themes
% used for structuring the document.
%
%%%%%%%%%%%%%%%%%%%%%%%%%%%%%%%%%%%%%%%%%

%----------------------------------------------------------------------------------------
%	PACKAGES AND OTHER DOCUMENT CONFIGURATIONS
%----------------------------------------------------------------------------------------

\documentclass[11pt,a4paper,times]{moderncv} % Font sizes: 10, 11, or 12; paper sizes: a4paper, letterpaper, a5paper, legalpaper, executivepaper or landscape; font families: sans or roman

\moderncvstyle{classic} % CV theme - options include: 'casual' (default), 'classic', 'oldstyle' and 'banking'
\moderncvcolor{grey} % CV color - options include: 'blue' (default), 'orange', 'green', 'red', 'purple', 'grey' and 'black'

%\usepackage{lipsum} % Used for inserting dummy 'Lorem ipsum' text into the template

%\usepackage[scale=0.8]{geometry} % Reduce document margins
%\usepackage[scale=0.8, top=1cm, bottom=1.7cm, headheight=11pt, headsep=12pt, footskip=12pt]{geometry}
\usepackage[scale=0.9]{geometry}
 \geometry{
 a4paper,
 total={170mm,257mm},
 left=21mm,
 right=21mm,
 top=21mm,
 bottom=21mm
 }
\renewcommand*{\sectionfont}{\large\bfseries}
%\setlength{\hintscolumnwidth}{3cm} % Uncomment to change the width of the dates column
%\setlength{\makecvtitlenamewidth}{10cm} % For the 'classic' style, uncomment to adjust the width of the space allocated to your name

%----------------------------------------------------------------------------------------
%	NAME AND CONTACT INFORMATION SECTION
%----------------------------------------------------------------------------------------
\title{\LARGE\bfseries Teaching Statement}  % \Huge, \LARGE, \Large, \large, \normalsize, \small, \footnotesize, etc.

\firstname{} % Your first name
\familyname{} % Your last name

% All information in this block is optional, comment out any lines you don't need
\address{Vatsalya Sharma }%{Center for Mathematical Plasma and Astrophysics, \\
%KU Leuven, Belgium, 3001}
%\mobile{(302) 584 3464}
%\phone{(000) 111 1112}
%\fax{(000) 111 1113}
\email{vatsalya.sharma@kuleuven.be}
%\homepage{staff.org.edu/~jsmith}{staff.org.edu/$\sim$jsmith} % The first argument is the url for the clickable link, the second argument is the url displayed in the template - this allows special characters to be displayed such as the tilde in this example
%\extrainfo{additional information}
%\photo[70pt][0.4pt]{pictures/picture} % The first bracket is the picture height, the second is the thickness of the frame around the picture (0pt for no frame)
%\quote{"A witty and playful quotation" - John Smith}
\usepackage{dirtytalk}
%----------------------------------------------------------------------------------------
\begin{document}
%\author{Vatsalya Sharma}
%\makecvtitle % Print the CV title
\makecvtitle
%----------------------------------------------------------------------------------------
%	EDUCATION SECTION
%----------------------------------------------------------------------------------------
\vspace{-10mm}
\hline
\vspace{2mm}
%\section{Teaching philosophy and motivation} % reduce it drastically. Reduce to 2 lines
%\textbf{Teaching, to me, is a continuous journey} of learning and growth. I firmly believe that teachers hold the \textbf{power to shape the next generation of students}. %I shall \textbf{ignite the young minds} with an interest in exploring and understanding subjects beyond mere academic requirements while creating leaders that actively contribute towards building a positive society.\\
%----------------------------------------------------------------------------------------


%\section{Teaching experience}
%I started participating in teaching and mentoring activities for undergraduate students during my Masters at IIT Hyderabad. I handled fluid mechanics lab as a teaching assistant (TA), where I taught theoretical fundamentals of fluid mechanics through lab projects
%My journey as a teacher began as a \textbf{junior Teaching Assistant} (TA) during my masters at IIT Hyderabad, where I \textbf{oversaw lab experiments} in Fluid Mechanics and Heat Transfer for undergraduate students. I would help students perform experiments and the evaluation was based on successful demonstration of the experiments followed by viva voice exam. The experience was very limited, but I accumulated 
%As a student at one of the Indian Institutes of Technology (IIT), where admission is based solely on a highly competitive all-India entrance exam, I studied in a multicultural environment. I did my Master's and Ph.D. at IIT Hyderabad, located in the Deccan region of India, which is far from my home in the Himalayas. The local Telugu language and script in this region are as distinct from my mother tongue, Hindi, as English is from Arabic, and the local culture vastly differed from what I was accustomed to. However, the environment at IITs and in India promotes understanding, acceptance, and appreciation of different cultures, religions, and languages. \textbf{These inclusive values are deeply embedded in my personal and professional life, and I am committed to upholding and promoting them within the community at UAT}.\\
My journey as a teacher began as a \textbf{junior Teaching Assistant (TA)} for undergraduate (UG) fluid mechanics and heat transfer lab during my masters at IIT Hyderabad. I got more involved in teaching during my PhD, where I worked as a teaching assistant (TA) in \textbf{\textit{Basic and Advanced Computational Fluid Dynamics} courses} for graduate and senior UG students. \textbf{These math-intensive courses required students to write C++ codes} for solving fluid mechanics problems like flow over a backward-facing step as a home assignment. As a TA, I graded assignments and exams, helped set the question paper, and scheduled doubt classes. When students under-performed on a particular concept in an exam or assignment, it indicated that they had not completely grasped it, so I took additional sessions to review it. In the end, I re-tested the students to check their understanding. I observed that their performance improved significantly in the course by following this approach, and since then, I have used this approach in every course I have taught.\\ 
During this time, I observed that the graduate students consistently outperformed UG students in the course, especially in computer assignments. Discussions with the UG students made me realize that their primary interest is using CFD to solve engineering and scientific problems through commercial and open-source CFD software. Therefore, after discussions with my department, I \textbf{introduced and taught a hands-on course: \textit{Numerical Modeling using CFD Tools}} at IIT Hyderabad. The course ran for four years, from 2016-19, and had almost $100$ UG students per year with a $3$ hour-long class per week. For the first three lectures, I covered the basics of finite volume discretization, meshing, pressure, and density-based algorithms, stability, and convergence calculations. In the remaining lectures, I introduced problem statements that were engaging for the students and provided hands-on experience in solving real-world problems.\\
\textbf{For example, in one lecture}, I taught nozzle design using CFD. I started with a refresher on nozzles and compressible flow theory, with appropriate derivations. Then, I introduced converging-diverging nozzles, starting with geometry and meshing. I guided them through setting up the problem and explained each option and its underlying CFD theory, linking it to concepts taught in my previous lectures. I also highlighted the differences in solution approach between this lecture and earlier topics, such as unsteady incompressible flow over a cylinder. The lecture would conclude with an assignment where students were tasked with altering parameters such as flow conditions and mesh distribution, then creating a report that assessed the impact of these changes on the solution, compared the CFD with analytical results, and a detailed description of the steps they followed. \textbf{My goal with these reports was to provide students with a guide they could reference in the future}. Occasionally, I would introduce subtle errors in the boundary conditions that resulted in incorrect outputs despite the software not flagging them. This approach aimed to make students understand the importance of applying correct physics and modeling, reinforcing that CFD software can produce flawed results if given flawed inputs. At the end of the semester, I invited students to form groups and either propose their projects or choose from a list of projects I provided. I evaluated the students based on their presentations and the depth of their knowledge and analysis.\\
As a TA, I worked with students from every corner of India. The teaching medium is English in all the IITs, but not everyone has studied in a school where the primary language of instruction is English. Usually, these students belong to the under-represented sections of society. \textbf{I prioritized reaching out to such students, offering after-hours assistance, and addressing doubts to ensure everyone could succeed}, which improved their grades and understanding of the subject. Many students also fear implicit bias in grading. I found that the fear can be significantly reduced by replacing the student's name with random numbers.\\
%I love motorsports, faithfully follow Formula One and MotoGP, and had always dreamt of creating a team that competes in motorsports competitions. When I arrived at IIT-Hyderabad, together with the students, \textbf{I created a team to design and build an all-terrain race car for the all-India SAE-BAJA competition}. Students learn to apply the curriculum subjects to solve real-world engineering challenges by participating in BAJA. The team members belonged to every corner of India, each bringing a unique perspective. As the founding Team Principal of IITH Racing, I mixed each team department with students from different cultural and socioeconomic backgrounds, which helped them appreciate and learn from each other's perspectives and develop technical skills to build a race car from scratch while strictly adhering to the competition rulebook. To ensure our project was fully supported, I negotiated funding, equipment, and track space with the university each year, securing the necessary resources while consistently delivering results. As the students from different backgrounds bonded while working together, our collective efforts culminated in remarkable success. \textbf{Over seven years of competing in the competition from $2013$ to $2020$, our team's dedication and hard work resulted in a top ranking in $2020$, where we emerged as the best-performing team among all the $14$ IITs and ranked $20$ out of $362$ teams nationwide. At IIT Bombay, I will actively work with the SAE-BAJA team to replicate the success of IITH Racing and in due time expand to electric BAJA and Formula Student.} \\
%At my postdoc in Belgium, I work with researchers from international backgrounds, each with different life experiences. This led to many discussions about the similarities and differences in how science is conducted across cultures. My experiences in India and Belgium have inspired me to think more broadly about the questions I could tackle as a researcher. I now see cross-cultural collaborations as a way to overcome communication barriers hindering scientific progress.\\
At present, I teach computational modeling of weakly ionized plasma as part of \textbf{\textit{Introduction to Plasma Physics} at KU Leuven} to a class of 15 graduate students. Based on student feedback, I hold additional classes for those interested in learning finite-volume CFD fundamentals, improving classroom engagement. At the end of the course, the students form groups to apply their classroom theory to analyze plasma generated during atmospheric entry and compare its properties with available data of astrophysical plasma. Recently, I enrolled and successfully completed special hands-on course for Post Ph.D. Scientists on Data-driven Fluid Mechanics offered at the Von-Karmann Institute of Fluid Mechanics (VKI), Belgium.\\
%change the courses to UAT
%My experiences have shaped my teaching philosophy and methodology. 
\textbf{Several courses at the department closely align with my research interests, making it easier for me to teach them}, such as Computational Methods in Thermal and Fluids Engg, Heat Transfer, Fluid Mechanics and Thermodynamics. 
As a faculty member, I would \textbf{introduce the High Temperature Gas Dynamics and Data-Driven Fluid Mechanics (DDFM) courses} at the UG and Graduate levels in the department.\\
The \textbf{DDFM course will focus on} data-driven and machine-learning tools leading advancements in model-order reduction, flow control, and data-driven turbulence closures. The course will start by introducing the scope of machine learning models in fluid dynamics, then introduce the classic tools from signal processing that are usually off-the-shelf solutions for many practical problems. Then, I will cover modal analysis, focusing on the methods for linear dimensionality reduction. The final part of the course will cover applying data-driven and machine-learning methods to fluid mechanics. Students will be given assignments to write their own small codes using the off-shelf libraries, giving them hands-on experience in developing and deploying the codes that theory covers in the classroom. I will take theory exams and a project based on applying classroom learning from the course. \textbf{This course will help the students adapt to the new machine learning methods they can further use in their research and other areas of interest}.\\
I will also introduce the \textbf{Introduction to Plasma Dynamics} course. I will teach the thermodynamic and chemical process of plasma formation, especially during atmospheric entry and in plasma ground facilities. It will be followed by discussions on the chemical and thermal non-equilibrium conditions in plasma, how high-temperature gas affects transport properties during re-entry, the relation between statistical and classical thermodynamics, which is relevant to the course. At the end of the course, everything taught until now will be brought together to show how the overall Navier-Stokes equation is modified to account for plasma. The exam will be a mix of theory and a hands-on project on writing codes to calculate high-temperature gas properties in one dimension under equilibrium and non-equilibrium flow conditions.
%theoretical exams, 
%I will also introduce the \textbf{High Temperature Gas Dynamics, and Hypersonic Flow Physics} courses. By understanding the complex processes in hypersonic flows and atmospheric entry \textbf{these courses will serve as the entry point for the students towards a career in space and hypersonic aerothermodynamics}. In these courses, I will teach the physics of flow at hypersonic flow. I will also discuss the influence of high temperatures on both inviscid and viscous flows and introduce the thermodynamic and chemical process of plasma formation, especially during atmospheric entry and in plasma ground facilities. It will be followed by discussions on the chemical and thermal nonequilibrium conditions in plasma, how high-temperature gas affects transport properties during re-entry, the relation between statistical and classical thermodynamics, which is relevant to the course, the kinetic theory, the evaluation of transport properties in plasma, and a brief introduction to radiative gas dynamics. At the end of the course, everything taught until now will be brought together to show how the overall Navier-Stokes equation is modified to account for plasma. The exam will be a mix of theory and a hands-on project on writing codes to calculate high-temperature gas properties in one dimension under equilibrium and non-equilibrium flow conditions. \\
%My experiences in India and Belgium have shaped my teaching philosophy and methodology, which I shall use to successfully guide my students to solve challenges inside and outside the classroom. 
%As a faculty member, I will combine elements of my experiences in India and Belgium to integrate diversity into my research and teaching. My multicultural background equips me with the cultural sensitivity and understanding needed to connect with the students. Through my lived experiences, I will create a space where students and researchers from all backgrounds feel valued and empowered to contribute their unique perspectives. 
%I am dedicated to promoting a culture of respect, curiosity, and resilience that mirrors UAT’s mission to advance equity, diversity, and inclusion. In this space, I will nurture future leaders who will champion these values within the UAT community and beyond, thus shaping a more inclusive world.
%My experiences have shaped my teaching philosophy and methodology, which I shall use to successfully guide my students to solve challenges inside and outside the classroom. 
\textbf{Concluding remarks : } Teaching is a continuous journey of learning and growth, where acquiring and imparting knowledge never ceases. My experiences have shaped my teaching philosophy and methodology, which I shall use to successfully guide my students to solve challenges inside and outside the classroom.


%As an Assistant Professor, I look forward to \textbf{teaching and introducing courses at Missouri S\&T} closely aligned with my research interest, such as gas dynamics, thermodynamics, CFD, hypersonic flow, and plasma dynamics, and magneto-hydrodynamics.
\end{document}
%Virginia Tech: tatement of Teaching - Discuss prior teaching experience, teaching approach, and future teaching interests, including specific efforts and future plans to support the success of all students through curriculum, classroom environment, and pedagogy. Include discussion of mentoring experiences and approach, including any past efforts and future plans to foster equitable and inclusive learning environments.
